\ifx\BASEMATHPREAMBLE\undefined
	\newcommand{\BASEMATHPREAMBLE}{}

	\ifx\NOPACKAGES\undefined
		\usepackage{float}
		\PassOptionsToPackage{hyphens}{url}\usepackage[breaklinks=true]{hyperref}
		\usepackage{breakurl}
		\usepackage{fancyhdr}
		\usepackage{commath}
		\usepackage{mathtools}
		\usepackage[style=ieee]{biblatex}
		\usepackage{amssymb}
		\usepackage{siunitx}
		\usepackage{marvosym}
		\usepackage{amsthm}
		\usepackage{asymptote}
		\usepackage{csvsimple}
		\usepackage{titlesec}

		\usepackage{systeme}
		\usepackage{cancel}
		\usepackage{gauss}
		\usepackage[numbered]{bookmark}
		\usepackage{mathrsfs}
		\usepackage{alltt}
		\usepackage[most]{tcolorbox}

		\ifdefined\SMALLMARGINS
		\ifx\MARGINS\undefined
		\newcommand{\MARGINS}{1.5in}
		\fi
		\usepackage[margin=\MARGINS]{geometry}
		\fi
		\usepackage{verbatim}


		\ifxetex
			\usepackage{anyfontsize}
			\usepackage{fontspec}
		\else
			\usepackage[utf8]{inputenc}
		\fi

		\ifdefined\LIBERTINE
			\usepackage[T1]{fontenc}
			\usepackage{libertine}
			\usepackage[scaled=0.96]{zi4}
			\usepackage[libertine]{newtxmath}
		\fi


		\usepackage{algorithm}
		\usepackage[noend]{algpseudocode}
		%  \usepackage[]{algorithm2e}


	\fi

	% theorems styling
	\ifx\THMCOUNTER\undefined
		\theoremstyle{plain}
		\newtheorem{theorem}{Theorem}
		\newtheorem{lemma}{Lemma}
		\newtheorem{corollary}{Corollary}
		\theoremstyle{definition}
		\newtheorem{define}{Definition}
	\else
		\theoremstyle{plain}
		\newtheorem{theorem}{Theorem}[\THMCOUNTER]
		\newtheorem{lemma}{Lemma}[\THMCOUNTER]
		\newtheorem{corollary}{Corollary}[\THMCOUNTER]
		\theoremstyle{definition}

		\newtheorem{define}{Definition}[\THMCOUNTER]

	\fi


	\theoremstyle{remark}
	\newtheorem*{remark}{Remark}
	\newtheorem*{example}{Example}

	% problem formatting

	\newcounter{currproblem}

	\newcommand{\problemsec}{\stepcounter{currproblem} \problem{\thecurrproblem}}
	\newcommand{\problem}[1]{\section{Problem {#1}}}

	\ifdefined\CROWDMARK
		\newcommand{\NEWPAGEPROBLEM}{} % crowdmark pagebreak
	\fi

	\ifdefined\NEWPAGEPROBLEM
		\newcommand{\sectionbreak}{\clearpage}
	\fi

	\ifx\EXCLUDEDEFAULTBIB\undefined
		\ifdefined\basedir
			\addbibresource{\basedir/bib/base.bib}
		\else
			\errmessage{Base repo directory for bib not found!}
		\fi
	\fi

	\ifx\NOFANCY\undefined
		\pagestyle{fancy}

		\fancyhead[L]{Assignment \assignmentnum}
		\fancyhead[R]{\coursename}
		\renewcommand{\footrulewidth}{0.4pt}


		\ifdefined\USESCSTYLE
				\fancyhead[C]{\textsc{\coursename, Assignment \assignmentnum}}
		\renewcommand{\headrulewidth}{0.0pt}
		\fancyhead[R]{}
		\fancyhead[L]{}
				\titleformat{\section}
		{\centering  \scshape}{\thesection.}{1em}{}
				\titleformat{\subsection}
		{\centering  \bfseries }{\thesubsection.}{.5em}{}
		\fi 
	\fi

	\newcommand{\integratedtitle}{%
	\ifx\assignmentcaption\undefined
	\newcommand{\assignmentcaption}{Assignment \assignmentnum}
	\fi
	{
	% \noindent \hrulefill%

	\noindent%
	\studentname \hfill \coursename%
	
	\noindent%
	\studentid, \ifdefined\studentemail \texttt{\studentemail} \fi
	\hfill \assignmentcaption%

	\noindent \hfill \today%

	}%
	\noindent \hrulefill
	\let\integratedtitle\undefined%
	\let\assignmentcaption\undefined%
	}%

	\newcommand{\solution}{\subsubsection*{Solution}}
	\newcommand{\parttitle}[1]{\subsection*{Part {#1}}}

	\ifdefined\USESOLUTIONBOX
	\newtcolorbox{solutionbox}[1][]{
	enhanced,
	arc=0pt,
	outer arc=0pt,
	colback=white,
	boxrule=0.8pt,
	#1
	}
	\else
	\renewenvironment{solution}{%
		\bigbreak%
		\noindent \textit{Solution.}%
	}{%
	\hspace*{0pt}\hfill\null$\blacksquare$%
	% \noindent\rule{\textwidth}{0.4pt}
	\bigbreak}
	\fi

	\newcommand{\intr}[2]{\int_{2}^{1}}
	\newcommand{\intba}{\int^b_a}
	\newcommand{\intzeroone} {\int_0^1}
	\newcommand{\diff}[2]{\frac{d{#1}} {d{#2}} }
	\newcommand{\pdif}[2]{\frac{\partial{#1}}{\partial{#2}} }
	\newcommand{\pdifs}[3]{\frac{\partial^{2}{#1}}{\partial{#2}\partial{#3}}}
	\newcommand{\scriptU}{\mathcal{U}}
	\newcommand{\scriptL}{\mathcal{L}}
	\newcommand{\scriptC}{\mathcal{C}}
	\newcommand{\scriptD}{\mathcal{D}}
	\newcommand{\defeq}{:=}


	\newcommand{\doubleR}{\mathbb{{R}}}
	\newcommand{\doubleN}{\mathbb{N}}
	\newcommand{\doubleZ}{\mathbb{Z}}
	\newcommand{\doubleC}{\mathbb{C}}
	\newcommand{\doubleQ}{\mathbb{Q}}
	\newcommand{\doubleO}{\mathbb{O}}
	\newcommand{\doubleF}{\mathbb{F}}
	\newcommand{\doubleA}{\mathbb{A}}
	\newcommand{\doubleP}{\mathbb{P}}
	\newcommand{\doubleD}{\mathbb{D}}

	\newcommand{\half}{\frac{1}{2}}
	\newcommand{\lgabs}[1]{\log \abs{#1}}
	\newcommand{\lnabs}[1]{\ln \abs{#1}}

	\newcommand{\lhospitals}{{L'H\^{o}pital's }}
	\newcommand{\sumtoinfty}[1]{\sum_{{#1}}^\infty }
	\newcommand{\abr}[1]{\left \langle {#1} \right \rangle}
	\newcommand{\flr}[1]{\left \lfloor {#1} \right \rfloor}
	\newcommand{\cil}[1]{\left \lceil {#1} \right \rceil}
	\renewcommand{\Re}{\operatorname{Re}}
	\renewcommand{\Im}{\operatorname{Im}}
	\newcommand{\Log}{\operatorname{Log}}
	\newcommand{\Arg}{\operatorname{Arg}}
	\newcommand{\lap}{\Delta}
	\newcommand{\grad}{\nabla}
	\newcommand{\dl}{\partial}
	\newcommand{\normal}{\hat{n}}
	\newcommand{\Aut}{\operatorname{Aut}}
	\newcommand{\Res}{\operatorname{Res}}
	\newcommand{\sinc}{\operatorname{sinc}}

	\makeatletter
	\newcommand{\curl}
	{\@ifstar {\nabla \times} {\operatorname{curl}}}

	\renewcommand{\div}
	{\@ifstar {\nabla \cdot}{\operatorname{div}}}

	\renewcommand{\diff}{\@ifstar {\dif} {\; \mathrm{d}}}

	\makeatother

	\newcommand{\sint}[4]{
	{
	\int_{#1} {#2} \dif y \wedge \dif z + {#3} \dif z \wedge \dif x + {#4} \dif x \wedge \dif y
	}
	}

	\newcommand{\cdif}{\cdot \dif}

	\newcommand{\len}{\ell}
	\newcommand{\gt}{>}
	\newcommand{\comp}{\circ}

	\newcommand{\PC}{\mathcal{PC}}

	% basic sets

	\newcommand{\complex}{\doubleC}
	\newcommand{\real}{\doubleR}
	\newcommand{\integer}{\doubleZ}
	\newcommand{\rational}{\doubleQ}
	\newcommand{\rationals}{\doubleQ}
	\newcommand{\naturals}{\doubleN}
	\newcommand{\field}{\doubleF}

	\newcommand{\conj}[1]{\overline{#1}}
	\newcommand{\powerset}{\mathcal{P}}
	\newcommand{\floor}{\operatorname{floor}}
	\newcommand{\ceil}{\operatorname{ceil}}
	\newcommand{\toquotation}{``$\Rightarrow$''\ }
	\newcommand{\fromquotation}{``$\Leftarrow$''\ }
	\newcommand{\func}[5]{{#1}:{#2} \to {#3}, \quad {#4} \mapsto {#5}}
	\newcommand{\Hess}{\operatorname{Hess}}
	\newcommand{\interior}{\operatorname{int}}
	\newcommand{\content}[1]{\mu\del{{#1}}}

	\newcommand{\COne}{\mathcal{C}^1}
	\newcommand{\sgn}{\operatorname{sgn}}
	\newcommand{\concat}{\oplus}

	\newcommand{\unitdisc}{\doubleD}

	\newcommand{\isom}{\cong}
	\newcommand{\btu}{\bigtriangleup}
	\newcommand{\symdiff}{\btu}

	% infinite series notation

	\makeatletter
	\newcommand{\diverges}
	{\@ifstar {\in \mathcal{D}} {= \infty}}

	\newcommand{\converges}
	{\@ifstar {\in \mathcal{C}} { < \infty}}

	\makeatother

	% linear algebra stuff

	\newcommand{\Col}{\operatorname{Col}}
	\newcommand{\Row}{\operatorname{Row}}
	\newcommand{\col}{\operatorname{col}}
	\newcommand{\row}{\operatorname{row}}
	\newcommand{\rank}{\operatorname{rank}}

	\newcommand{\Trace}{\operatorname{Trace}}
	\newcommand{\ord}{\operatorname{ord}}
        \newcommand{\Mat}{\operatorname{Mat}}
        \newcommand{\GL}{\operatorname{GL}}

	\makeatletter
	\newcommand{\Null}
	{\@ifstar {\operatorname{n}} {\operatorname{null}}}
	\newcommand{\trace}
	{\@ifstar {\operatorname{tr}} {\operatorname{trace}}}
	\makeatother
	\newcommand{\nullspace}{\Null}

	\newcommand{\End}{\operatorname{End}}
	\newcommand{\Id}{\operatorname{Id}}

	\newcommand{\dsum}{\oplus}
	\newcommand{\cyc}{\abr}
	\newcommand{\bigdsum}{\bigoplus}

	\newcommand{\la}{\leftarrow}
	\ifdefined\nequiv
	\else
		\newcommand{\nequiv}{\not\equiv}
	\fi

	\newcommand{\imag}{i}
	\newcommand{\exchange}{\leftrightarrow}
	\newcommand{\setspan}{\operatorname{span}}
	\newcommand{\superset}{\supset}

	\newcommand{\Or}{\;\operatorname{or}\;}
	\renewcommand{\And}{\;\operatorname{and}\;}
	\newcommand{\lra}{\leftrightarrow}

	\newcommand{\contradiction}{\Lightning}
	\newcommand{\rlightning}{\begin{flushright} \Lightning \end{flushright}}

	\newcommand{\ms}{\scriptstyle}

	% stastical stuffs

	\newcommand{\EV}{\mathbb{E}}
	\newcommand{\Var}{\operatorname{Var}}
	\newcommand{\V}{\Var}
	\newcommand{\Cov}{\operatorname{Cov}}
	\newcommand{\from}{\mid}

	% diff equations

	\newcommand{\laplace}{\mathscr{L}}

	% alg.geometry
	\newcommand{\affine}{\doubleA}
	\newcommand{\Jac}{\operatorname{Jac}}
	\newcommand{\Rad}{\operatorname{Rad}}
	\newcommand{\Spec}{\operatorname{Spec}}
	\newcommand{\Frac}{\operatorname{Frac}}
	\newcommand{\proj}{\doubleP}
	\newcommand{\im}{\operatorname{im}}
	
	% shorthand for blackboard letters
	\newcommand{\bbR}{\doubleR}
	\newcommand{\bbQ}{\doubleQ}
	\newcommand{\bbZ}{\doubleZ}
	\newcommand{\bbC}{\doubleC}
	\newcommand{\bbN}{\doubleN}

	% arg-stuff
	\newcommand{\argmax}{\operatorname*{argmax}}
	\newcommand{\argmin}{\operatorname*{argmin}}
	\newcommand{\Hom}{\operatorname{Hom}}
	\newcommand{\negate}{\operatorname{\sim}}

	\ifdefined\DONOTDEFINE
		\errmessage{DONOTDEFINE should not be defined}
	\fi

\fi
