Consider this problem: You are in charge of planning the city's sewage network. 
You have the city core that produces the sewage, several pump stations located on the outskirts, and the water treatment plant in another town. 
Your task is to maximize the amount of sewage from the city to the water treatment plant.
How do you it?

\section{Defining The problem}

\begin{define}
    A ``flow network'' is a tuple of a graph $G=(V,E)$, the source and sink $s,t\in V$ and the capacity function $c: E\to\real_{\ge 0}$. 
    The sources provide flow into each edge which the sink recieves.
\end{define}

In other words, from the example above the vertices would be the city as the source, pumping stations and the treatment plant as sink, the edges as water pipes, and the capacity as the maximal amount water can flow in a given pipe. 
With this, we can define a criteria for all solutions to this problem:

\begin{define}
    \label{ch3:flow-def}
    A ``flow'' is a particular solution to a flow network $(G=(V,E),s,t,c)$ which is represented as a function $f: E \to \real_{\ge 0}$ such that for all $(u,v)\in E$, 

    \begin{equation}
        0 \le f((u,v)) \le c((u,v))
    \end{equation}

    i.e. the flow from one node to another cannot exceed the capacity of that edge (e.g. pipe)
    and for all $v \neq s, t$, 

    \begin{equation}
        \sum_{(u,v) \in E} f((u,v)) = \sum_{(v,u) \in E} f((v,u))
    \end{equation}

    i.e. the flow entering node $v$ must be the same as flow exiting $v$.
    $v$ cannot generate or hoard any flow for its own use. 
\end{define}

Note that we can impose additional restrictions to make the problem a bit more simple With

\begin{enumerate}
    \item $G$ is a directed graph;
    \item There are no self loops (i.e. no $(u,u)\in E$ for any $u\in V$); 
    \item There are no bidirectional edge pairs (i.e. if $(u,v)\in E$, then $(v,u) \notin E$).
\end{enumerate}

Moreover, with $c:E\to\real_0$, for a simpler process we can write $c$ as a function from $V^2$ (i.e. from a pair of edges to a nonnegative $\real$) as

\begin{equation}
    c: V^2 \to \real_{\ge 0}, \quad (u,v) \mapsto \begin{cases}
        c((u,v)), & (u,v)\in E \\
        0,& \text{otherwise}
    \end{cases}
\end{equation}

Note that if $G$ doesn't satisfy these assumptions above, we can ``massage'' $G$ to fit our assumption by

\begin{enumerate}
    \item If $G$ is undirected, let $G$ instead have two directed edges with the same capacity; 
    \item If $G$ has self-loops, just remove them as self-loop contributes nothing to the overall flow;
    \item If $G$ has bidirectional edge pairs, let $(u,v)\in E$ as it is, and for $(v,u)$, we route it through a proxy node instead.
    This means creating a node $z \in V$ and letting $(v,z)$ and $(z,u)$ with the capacity as $(v,u)$.
    The end result will be equivalent. 
\end{enumerate}

\section{Defining the Objective}

Now, as with the max-flow problem, our goal is to maximize the flow into $t$, or the sink.
In the sewage example, this means processing wastewater in the most efficient manner possible (i.e. most \si{\liter\per\hour} of sewage).

Na\"ively, we can define the goal as 

\begin{equation}
    \label{ch3:naive-obj}
    \max_{f} \sum_{u\in V} f(u,t)
\end{equation}

where $f$ is a flow subject to the defintion \ref{ch3:flow-def}. 
However, problem arises that if $t$ has an outgoing edge to another node, then there is a path back to $t$, we can artifically inflate the flow without gaining a \textit{net} flow. 
Consider the case of

\begin{figure}[H]
    \centering
    \asyinclude{asy/ch3-maxflow-naive-t.asy}
    \caption{Where the na\"ive objective fails }
\end{figure}

From the above, we could set $f(t,a)=1000=f(a,b)=f(b,t)$. 
We have artifically inflated the flow amount by $1000$ for doing absolutely nothing. 
Moreover, this is a well-defined flow. What can we do? The na\"ive method is almost there - it's just missing one little detail - to consider the \textit{net} flow instead of the \textit{gross} flow. 
We can modify objective \ref{ch3:naive-obj} into

\begin{equation}
    \label{ch3:naive-obj-fixed}
    \max_f \del{ \sum_{u\in V} f(u,t) - \sum_{u \in V} f(t,u) }
\end{equation}

With this, we only consider the total flow amount in, minus the flow from $t$. 
Still, to make the definition a little more simple, we can alternatively formulate the outflow by considering the net outflow of $s$, or formally:

\begin{define}
    Let $f$ a valid flow. 
    Then the ``flow value'' of $f$, denoted $\abs{f}$ is defined as 

    \begin{equation}
        \abs{f} \defeq \sum_{v \in V} f(s,v) - \sum_{v \in V} f(v,s)
    \end{equation}
\end{define}

With flow conservation property, we can construct a telescoping sum that cancels out to \ref{ch3:naive-obj-fixed}. 

\begin{lemma}
    Let $f$ a valid flow. Then 

    \[
        \abs{f}=\sum_{u\in V} f(u,t) - \sum_{u \in V} f(t,u). 
    \]
\end{lemma}

\begin{proof}
    It is clear that 

    \[
        \abs{f} = \sum_{v \in V} f(s,v) - \sum_{v \in V} f(v,s).
    \]

    By flow conservation, for each $u \in V \setminus \cbr{s, t}$, 

    \[
        \sum_{v \in U} f(u,v) = \sum_{v \in U} f(v,u)
    \]

    We can sum up for all $u$. Hence, 

    \[
        \sum_{u \in V\setminus\cbr{s,t}} \sum_{v \in U} f(u,v) =  \sum_{u \in V\setminus\cbr{s,t}} \sum_{v \in U} f(v,u)
    \]

    The left hand side is 

    \[
        \sum_{u \in V\setminus\cbr{s,t}} \sum_{v \in V\setminus \cbr{s,t}} f(u,v) + \sum_{u \in V\setminus\cbr{s,t}} \del{ f(u,s) + f(u,t)}
    \]

    The right hand side is 

    \[
        \sum_{u \in V\setminus\cbr{s,t}} \sum_{v \in V\setminus \cbr{s,t}} f(u,v) + \sum_{u \in V\setminus\cbr{s,t}} \del{ f(s,u) + f(t,u)}
    \]

    Subtract the remaining sum from other sides; we then have 

    \[
        \sum_{u \in V\setminus\cbr{s,t}} \del{ f(u,s) + f(u,t)} = \sum_{u \in V\setminus\cbr{s,t}} \del{ f(s,u) + f(t,u)}
    \]

    Which means 

    \[
        \sum_{u \in V\setminus\cbr{s,t}} f(s,u) - \sum_{u \in V\setminus\cbr{s,t}} f(u,s) = \sum_{u \in V\setminus\cbr{s,t}} f(u,t) - \sum_{u \in V\setminus\cbr{s,t}} f(t,u)
    \]

    Then, we can first add $f(s,s)-f(s,s)=0$ to the left hand side, $f(t,t)-f(t,t)=0$ on the right hand side, and $f(s,t)-f(t,s)$ to both sides, we have 

    \[
	\abs{f}=\sum_{u \in V} f(s,u) - \sum_{u \in V} f(u,s) = \sum_{u \in V} f(u,t) - \sum_{u \in V} f(t,u)
    \]

    which is the same as objective \ref{ch3:naive-obj-fixed}. 
\end{proof}

\section{The Ford-Fulkerson Method}

Note that instead of the term ``Algorithm'', we use ``Method'' since this can be implemented in various of ways. 
Here is a rough sketch of method: 

\begin{algorithm}[H]
    \caption{The Ford-Fulkerson Method}
    \label{ch3:ford-fulkerson}

    \begin{algorithmic}[1]
        \Procedure{Ford-Fulkerson}{$V,E , s, t, c$}
        \State $f\gets$ an zero flow function \Comment i.e $(u,v)\mapsto 0$ for all $u,v\in V^2$
        \While{There exists a path $s\to\ldots\to t$, call this path $P$ such that for all $(u,v)\in P$, $f(u,v)<c(u,v)$. }
        \State $\Delta\gets \min_{(u,v)\in P} \del{c(u,v)-f(u,v)}$ 
        \State Augment $f$ on $(u,v)\in P$ with $\Delta$ \Comment i.e. $f(u,v)\gets f_{\text{old}}(u,v)+\Delta$ for $(u,v)\in P$. 
        \EndWhile

        \State \Return $f$ 
        \EndProcedure
    \end{algorithmic}
\end{algorithm}