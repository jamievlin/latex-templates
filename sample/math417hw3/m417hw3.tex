
% As a sample LaTeX document, this is an actual assignment 
% written in LaTeX with my template for MATH 417, 
% Honors Real Variables (Measure Theory) at University of Alberta.
% This source has been released with permission with the instructor,
% Professor John C. Bowman as the solutions are available at
% https://www.math.ualberta.ca/~bowman/m417/assign3.pdf

% ------- END DISCLAIMER ------------- 

% https://www.math.ualberta.ca/~bowman/m417/assign3.pdf
\documentclass{article}
\usepackage{import}

% For the meaning behind these commands, see math417hw1/m417hw1.tex

\newcommand{\SMALLMARGINS}{1.5in}
\newcommand{\basedir}{../../}
\newcommand{\USESCSTYLE}{}

% \newcommand{\CROWDMARK}{}

\import{\basedir/base/}{mathPreamble}

\newcommand{\assignmentnum}{3}
\newcommand{\coursename}{MATH 417}

\newcommand{\studentname}{Supakorn ``Jamie'' Rassameemasmuang}
\newcommand{\studentid}{1234567}
\newcommand{\studentemail}{supakorn.jamie@ualberta.ca}
\newcommand{\coursesec}{Q1}


\begin{document}

\integratedtitle

\problemsec

What is the wrong in the following argument that \textit{The set $(0,1)$ is countable}?
Express the open set $(0,1)$ as a union of open intervals $I_x$ centered on each real $x\in (0,1)$:
\[
    \bigcup_{x\in (0,1)} I_x=(0,1).
\]
Each interval $I_x$ contains a rational which we can use as a label for that interval.
Since there are only countably many rational numbers, there are countably many labels.
Hence the above union actually contains only a countable number of intervals, each centered at a real number.
There are therefore only a countable number of real numbers.

\begin{solution}
    Solution removed.
\end{solution}

\problemsec 

Let $S\subset \real^d$ such that
\[
    m^*(E\cap S) + m^*(E\cap S^c) = m(E).
\]
for all elementary set $E$.
Show that $S$ is Lebesgue measurable.

\begin{solution}
    Solution removed.
\end{solution}

\problemsec

Suppose $S_n\subset \real^d$ for $n\in \naturals$ are Lebesgue measurable sets that converge pointwise to a set $S$.

\subsection{Part (a)}

Show that $S$ is Lebesgue measurable.

\begin{solution}
    Solution removed.
\end{solution}

\subsection{Part (b)}
\textit{Dominated Convergence}.
Suppose that $S_n$ are all contained inside another Lebesgue measurable set $F$ of finite measure.
Show that $m(S_n)\to m(S)$ as $n\to \infty$.

\begin{solution}
    Solution removed.
\end{solution}

\subsection{Part (c)}

Give a counterexample to show that the dominated convergence theorem fails if the $S_n$ are not contained in a set of finite measure, even if we assume that $m(S_n)$ are all uniformly bounded.

\begin{solution}
    Solution removed.
\end{solution}

\problemsec

Show that an unsigned function $f:\real^d\to[0,\infty]$ is a simple function if and only if it is measurable and takes on finitely many values.

\begin{solution}
    Solution removed.
\end{solution}

\problemsec

Let $d,d'\in \naturals$.

\subsection{Part (a)}

If $A\subset \real^d$ and $B\subset \real^d$, show that
\[
    m^{d+d'*}(A\times B)\le m^{d*}(A)m^{d'*}(B).
\]
where $m^{d*}$ denotes the $d$-dimensional Outer Lebesgue Measure.

\begin{solution}
    Solution removed.
\end{solution}

\subsection{Part (b)}
Let $A\subset \real^d$ and $B\subset \real^{d'}$ Lebesgue measurable sets (not necessarily of finite measure).
Show that $A\times B$ is Lebesgue Measurable with
\[
    m(A\times B)=m(A)m(B).
\]

\begin{solution}
    Solution removed.
\end{solution}

\problemsec

If $f:\real^d\to [0,\infty]$ is Lebesgue measurable, show that the Lebesgue measure of
\[
    A_f \defeq \set{(x,y)\in \real^{d+1} : 0\le y \le f(x)}
\]
exists and equals $\int_{\real^d} f$.

\begin{solution}
    Solution removed.
\end{solution}
\end{document}